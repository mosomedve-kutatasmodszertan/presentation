\begin{frame}{Interpretation of Results}
\begin{itemize}
    \item Clear reduction in hallucination rates across all tested domains.
    \item Both \textbf{data quality improvement} and \textbf{architectural refinement} proved effective.
    \item Comparative analysis shows:
    \begin{itemize}
        \item Data reliability improvements help certain domains more.
        \item Architectural optimization benefits others more.
    \end{itemize}
    \item Indicates complementary effects – neither aspect alone is sufficient.
    \item \textbf{Conclusion:} Robust factual consistency requires a holistic approach integrating both data- and architecture-centric strategies.
\end{itemize}
\end{frame}

%-----------------------------------------------------------

\begin{frame}{Limitations}
\begin{itemize}
    \item \textbf{Benchmarking challenges:} Current tools (e.g., HaluEval 2.0) cannot fully capture nuanced hallucinations.
    \item Automated factuality checks introduce measurement uncertainty.
    \item \textbf{Data constraints:} 
    \begin{itemize}
        \item High-quality data often excludes low-reliability sources.
        \item Yet such sources may contain unique, valuable knowledge.
    \end{itemize}
    \item Balancing data inclusion vs.\ factual stability remains unresolved.
\end{itemize}
\end{frame}

%-----------------------------------------------------------

\begin{frame}{Future Directions}
\begin{itemize}
    \item Develop more comprehensive hallucination benchmarks
    \begin{itemize}
        \item Integrate factual and contextual dimensions
    \end{itemize}
    \item Scale experiments to larger models and diverse domains
    \item Explore hybrid retrieval-augmented architectures
    \item Deepen understanding of how data and structure interact in hallucination suppression
\end{itemize}
\begin{block}{Goal}
    Toward more \textbf{trustworthy, knowledge-grounded} generative models.
\end{block}
\end{frame}
